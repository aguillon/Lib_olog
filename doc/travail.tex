\documentclass[a4paper]{scrartcl}

% Compile avec `xelatex travail.tex`

\usepackage{polyglossia}
\setdefaultlanguage{french}
\usepackage{unicode-math}
\usepackage{fontspec}

\setmainfont[Ligatures=Rare]{Linux Libertine O}



\usepackage[backend=biber,style=authoryear-comp,uniquename=init,firstinits=true,
            uniquelist=false,maxcitenames=2,mincitenames=1,maxbibnames=99,
            isbn=false,url=false,doi=false
]{biblatex}

\renewcommand{\cite}{\parencite}
\renewcommand*{\nameyeardelim}{\addcomma \addnbspace}
\renewcommand*{\revsdnamedelim}{}
\renewcommand*\finalnamedelim{ \& }

\DefineBibliographyExtras{french}{\restorecommand\mkbibnamelast}

\DeclareNameAlias{default}{last-first}
\DeclareNameAlias{sortname}{last-first}

% Supprimer les guillemets dans les titres
\DeclareFieldFormat[article,incollection,unpublished,inproceedings]{title}{#1}

\usepackage{xpatch}
\usepackage{xstring}

\renewbibmacro*{series+number}{%
  \setunit*{\addnbspace}%
  \printfield{series}%
  \printfield{number}%
  \newunit}

\xpatchbibmacro{textcite}{\addcoma}{}{}{}

\addbibresource{./references.bib}


\begin{document}

\title{Une bibliothèque pour la manipulation d'ontologies formelles}
\subtitle{Document de travail}

\maketitle

\abstract{Ce document présente les fonctionnalités à implémenter dans la
  bibliothèque, ainsi que des pistes de développement orientées vers le
  \emph{data mining}. Bien que ce projet soit avant tout un projet scolaire, les
  contributions extérieures sont les bienvenues.}

\section{Introduction}

La représentation des connaissances est un domaine de l'Intelligence
Artificielle visant à structurer des informations sur le monde pouvant être
utilisées dans des applications de haut niveau. Ces représentations peuvent
ensuite être visualisées, comparées entre elles ou encore utilisées pour le
raisonnement automatique \cite{brachman-knowledge-representation}.

\textcite{spivak-ologs} introduisent les «\emph{ontology logs}» (ou « ologs »),
une nouvelle représentation des connaissances qui se base sur la théorie des
catégories pour structurer les ontologies. Celles-ci sont décrites sous forme de
types (objets) et d'aspects (flèches entre les objets) et font intervenir les
constructions usuelles des catégories : diagrammes commutatifs (égalité entre
deux aspects), produits, etc.

La présente bibliothèque vise à proposer une implémentation simple des ologs
qui sera par la suite appliquée à la fouille de données.

\section{Fonctionnalités}

La bibliothèque devra proposer les fonctionnalités suivantes :
\begin{itemize}
  \item représentation, enregistrement et chargement d'ologs ;
  \item fonctions génériques permettant la création et la modification
    d'ontologies en OCaml aussi simplement que possible ;
  \item population d'ontologies simples et vérification automatisée du respect
    des «faits» déclarés par l'olog ;
  \item affichage des ontologies dans un format graphique à définir ;
\end{itemize}

Par la suite, des fonctionnalités avancées pourront être implémentées, telles
que l'export vers SQL, la génération de code OCaml (façon \emph{type
providers}) pour la «programmation orientée ontologies» ou l'intégration d'un
prouveur.

\section{Rappel sur les ologs}

\section{Application au \emph{data mining}}

\subsection{Littérature}

Dans un second temps, la bibliothèque sera appliquée à des problèmes de fouille
de données (\emph{data mining}). Bien que leur rôle premier soit de représenter
des connaissances structurées par l'utilisateur même, les ontologies ont déjà
été appliquées par plusieurs auteurs à différentes tâches de la fouille de
données telles la recherche de motifs ou le clustering.

Une application récurrente des ontologies au clustering consiste à raffiner les
distances et méthodes de pondération utilisées pour tenir compte d'informations
sémantiques et améliorer les performances du clustering, ou au contraire tenir
compte de la subjectivité des utilisateurs. Par exemple en clustering de
documents textes, \textcite{hotho-ontology-clustering} utilisent un graphe
d'ontologies (une hétérarchie plutôt qu'une hiérarchie) afin de trouver des
concepts adaptés à la description des différents clusters.

L'approche par olog doit également être comparée aux \emph{topic maps},
présentées par exemple dans \textcite{ellouze-topic-maps-thesis} (voir par
exemple p. 36, Figure 2.3).

\printbibliography

\end{document}
